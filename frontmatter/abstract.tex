\providecommand\phantomsection{} \phantomsection
%\addcontentsline{toc}{part}{Abstract}
\thispagestyle{plain}
\begin{center}
%\hrule
\providecommand\pdfbookmark[3][]{} \pdfbookmark[0]{Abstract}{bm:Abstract}
% \vspace*{1in}
\textbf{ABSTRACT}\\[2\baselineskip]
% \vspace*{.1in}
\end{center}

Determining stellar parameters using high-precision asteroseismic constraints has improved our understanding of exoplanetary systems and galactic history. With such advances comes scrutiny of systematic uncertainties in our stellar models. In this report, I present a new method for determining fundamental stellar properties using a hierarchical Bayesian model. The hierarchical model enables the addition of model parameters, to address assumptions usually made when modelling stars, by encoding population-level information into a sample of stars. I present the first application of this method to a sample of low-mass dwarf stars (\SIrange{0.8}{1.2}{\solarmass}) and use it to infer the population distribution of helium ($Y$) and mixing-length theory parameter ($\alpha_\mathrm{mlt}$). I find that the hierarchical model can obtain stellar ages, masses and radii to good precision despite the addition of uncertainty from $Y$ and $\alpha_\mathrm{mlt}$. Finally, I discuss a possible extension to the model which makes use of asteroseismic signatures of helium abundance to further constrain helium.
