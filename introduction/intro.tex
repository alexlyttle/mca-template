\chapter{Introduction}

\section{Astroseismology}

For over a century, we have been able to map stars based on their photometric magnitude and spectroscopic colour using Hertzsprung-Russell (HR) diagrams. Coupling such observational data with measurements of interstellar distances using parallax, we were able to determine stellar luminosities. The unique structure of early HR diagrams eluded to the idea that stars evolve over time. With the addition of nuclear physics, theories of stellar evolution could be put to the test. However, while we could only observe stellar surface properties, many modelling mysteries would be left unsolved.

Until the last few decades, our understanding of stellar structure has been all but skin deep. In the 1960s, observations of 5-minute brightness fluctuations in the solar photosphere lead to the study of stochastically driven acoustic waves trapped beneath the surface of the Sun \citep{Ulrich1970, Ando.Osaki1975}. Later named helioseismology \citep{Deubner.Gough1984}, the study of oscillation modes allowed for further insights into the solar interior, such as rotation \citep{Deubner.Ulrich.ea1979} and solar neutrino production \citep{Bahcall.Ulrich1988}. In tandem with this research was the emergence of asteroseismology -- the study of stars through their oscillation frequencies \citep{Christensen-Dalsgaard1984}.

Give examples of the sorts of things asteroseismology can help us uncover, from ages \citep[see, e.g.]{Ulrich1986, Soderblom2010, SilvaAguirre.Davies.ea2015} to masses and radii from scaling relations \citep{} and fitting stellar models\citep{}.

\subsection{Solar-like oscillators}

Solar-like oscillators are stars for which their acoustic oscillator modes (or p modes) are excited stochastically by convection in their outer layers.

Review some work on solar-like oscillators and fundamental parameters.

\subsection{Detecting oscillation modes}

Name some missions which were able to detect asteroseismic oscillations and review their limitations.

Explain the types of stars which can be observed and methods we can obtain the oscillations

\subsection{Helium II Ionization Zone Glitch}

Explain what the glitch is an why it is useful.

Review work which uses this to get stellar parameters.

\section{Machine Learning in Astrophysics}

Review works which use machine learning in astrophysics, particularly asteroseismology.

\section{Hierarchical Bayesian Models in Astrophysics}

Review work with uses HBMs in astrophysics.

\section{Open-Source Code in Astrophysics}

Review the state-of-the-art of open-source code in astrophysics.
