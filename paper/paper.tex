\chapter{Hierarchically Modelling Many Stars}\label{ch:paper}

I have demonstrated how an HBM can improve inference with a simple example in Section \ref{sec:hbm}. In Section \ref{sec:models} I showed how we can use stellar evolutionary models to predict observables, and with the addition of asteroseismology (Section \ref{sec:seismo}) improve constraints on the internal structure and average properties of a star. Now, I will show how these methods can be combined to hierarchically model many stars.

The paper accompanying this report (Lyttle et al., in prep.) is attached in Appendix \label{apx:paper}. The paper in its current state is being prepared for submission, awaiting feedback and discussion from co-authors. In this paper, I report the first application of our new method for hierarchically modelling many stars to a sample of dwarfs and subgiants observed with \emph{Kepler}.
